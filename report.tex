\documentclass[a4paper, 11pt]{article}
\usepackage{comment} % enables the use of multi-line comments (\ifx \fi) 
\usepackage{lipsum} %This package just generates Lorem Ipsum filler text. 
\usepackage{fullpage} % changes the margin

\begin{document}
%Header-Make sure you update this information!!!!
\noindent
\large\textbf{Project Report} \hfill \textbf{Katie Swanson} \\
\normalsize CMSC 471 \hfill Teammates: Katie Dillon, Sasha (Alexander Bauer)\\
Prof. Max \hfill (worked next to them, got some l\
ines of code from them) \\
 \hfill Due Date: 05/04/16

\section*{Project 3}
For project 3 we had to write our own program to use image and feature recognition to classify images we pass to the progrm.

\section*{Investigation/Research}
In class, we learned about featurization and image classification. We also learned about training a machine learning program by using a training set, a testing set, and a validation set. I first knew that I would be using that general approach, but I did not know what libraries or code to use, or even how to start off. I asked my two friends, Katie and Sasha, to work on the project in the same room as me. Being a mac user, and them two being linux users, they got way ahead of me because I had a lot of problems installing the programs initially. But, we researched ways to approach this problem, and cv2 sounded like a good library to read in files etc. and sklearn sounded like a good library to help with the machine learning and predictions. Eventually, I did a lot of work myself, although I will admit I did get some help from Katie in the end.

\section*{Approach}
I used cv2 to help me read in the files in for loops for each of the five folders for each of the types of drawings. I then added the label to the label list and used the python imaging library (Pillow) to read in the image as RBG values. I then flattened this 2d array into one value by adding them all up. Then, I appended this value to the descriptors list. At the end, we have one huge descriptor and one huge label list, both of them matching so if you indexed [i] you would get the matching label of i in the descriptor list. After this, I used sklearn to create a model and to pass in my model, descriptors, and labels in order to train my set and print out my accuracy. sklearn does a lot of the work for me, which is a good thing since the ML part of this project is extremely complex. After that, I pickled my program using cv2. I then ran my pickled program, and when you enter a jpeg image in that saved instance of my trained set, it prints out what it thinks the image is classified under.

\section*{Pickling}
I pickled my trained data set into a file called 'proj3.p'. This is to be more efficient and not to waste time or resources retraining the set each time you run the program. You can run a file of type '.jpg' on the command line with my second python file that was written just to read in the .jpg file you specify and compare it to the pickled data set and make a prediciton based off of it. You can do this by typing 'python2.7 loadImg.py [filename.jpg]' into the terminal where the files are located, including the .jpg file. This new python file reads the pickled file where the trained data set is located, and compares the new .jpg file it has read in (its RGB value 2d array) to the data set, then makes a prediction based on the labels that matches with the descriptors that are closest to the new .jpg's descriptor values.

\section*{Accuracy}
My accuracy should currently be at around 93 percent.

\section*{Final Words}
I did work with two other people, so if our code is similar, we did not copy off of each other. We worked in the same room and happened to choose the same technologies since we wanted to help each other on the project.

\section*{Final Evaluation}
Overall, I learned so much more about ML and AI by applying the concepts we learned in class to programming an actual project. It got me excited to try out becoming an AI software engineer in the future.

\begin{thebibliography}{9}
\bibitem{Layout}https://www.overleaf.com/5108576cvqjbp#/15899417/
\end{thebibliography}

\end{document}
